%!TEX root = doe.tex

\section{Proxies and Power}
\label{sec:proxies-power}

Prior to developing our proxy, we build on our discussion in the Introduction regarding the fundamental choices underlying our approach.
Taken together, these points triangulate our general argument about what we mean by power and how to measure it.

\subsection{Why Dispute Outcomes?}

Like many, though not all, contemporary scholars, we orient our understanding of war in terms of the bargaining model.
Bargaining models require disagreement points, and in the context of international crises, disagreement means conflict.
The outcome of bargaining---whether an agreement is reached and, if so, which side it favors---depends on the likelihood of each potential outcome of fighting and the associated costs.
As we discussed in the Introduction, the distribution of outcomes is usually pinned down through a single exogenous parameter, $p$, which captures the probability that one country defeats the other in case of war.
In some of the most influential extant work on power \citep[e.g.,][]{powell1999}, $p$ is its only analytic representative.

One of the bargaining model's most powerful features is its provision for $p$ to influence \emph{peaceful} outcomes, despite its inherently conflictual character.
As \citet[3]{schellingpower} notes, ``it is the \emph{threat} of damage \ldots\ that can make someone yield or comply.''
In other words, the expected outcome of war sets the location of the bargaining range.
This, in turn, is the reason that stronger states enjoy better peaceful settlements \citep{banks1990equilibrium}.
Consequently, we feel comfortable taking victory in a dispute as an indicator of greater power even if the dispute did not proceed all the way to war.

Of course, the power to win hypothetical disputes is but one kind of power that states can exert over one another.
There are other outcomes that might be relevant, and there may be ways that states influence one another that are not related to dispute outcomes.
It is important to state explicitly our restriction in scope, but at the same time, this restriction is quite common for theorists and empiricists alike, especially those working within the bargaining paradigm.

\subsection{Why Material?}

Material capabilities are the starting point for much of what we know about power.
Military historians have accrued impressive amounts of information on states' material holdings \citep[e.g.,][Chapter~1]{taylor}, and realists have subsequently assigned materiel pride of place among explanators of power.
As we noted above, material measures of power are also important to liberals and constructivists, if only for the sake of clearly ruling out realist accounts.
The material conception of power has also served as a foil for those concerned with the particulars of force deployment or tactics more broadly \citep{biddle}.

More pragmatically, empirical scholars frequently use material capabilities, and particularly the ratio of CINC scores, as the data for their power proxies.
Examining publications from 2005 to 2014 in five top journals for empirical international relations research,\footnote{%
\label{fn:journals}
  \emph{American Political Science Review}, \emph{American Journal of Political Science}, \emph{Journal of Politics}, \emph{International Organization}, and \emph{International Studies Quarterly}.
}
we found at least 94 articles that control for the capability ratio or other proxies based on CINC scores.
Though many of these articles' main models included other measures for channels of influence from one state to another like alliances or investment, it remains remarkable that such a broad swath of articles would include a measure of material capabilities.
This is especially so because they cover a wide range of dependent variables, from conflict onset (easily the most common) to violations of international law to river treaty formation.
Our material approach to power, then, is of relevance to scholars across many areas of international relations.

By restricting our attention to the material dimensions of relative power, we also allow for an apples-to-apples comparison between our new measure and traditional CINC-based measures.
Had we incorporated new covariates into our new measurement approach, it would be difficult to assess whether any gains (or losses) were due to the modeling strategy or the additional data.
Doing so would also complicate the replication analysis we use to validate the new measure.
Since all of the aforementioned studies include CINC scores or some function thereof in their regressions, we can proceed assuming that the material capability components are not endogenous, post-treatment, or otherwise unwise to include.
If our measure included factors like alliance relationships or regime types, the universe of studies we could replicate to validate our approach would shrink considerably.

\subsection{Why Prediction?}

Our predictive criterion for measurement carries its own set of commitments.
We adopt this criterion because it prioritizes models that generalize well---those that avoid the underfitting of \emph{a priori} approaches and the overfitting of too-flexible approaches.
The risk of overfitting is particularly high when there are too many degrees of freedom relative to the amount of data available.
If the outcome of interest is only rarely observed, it might be hard to separate signal from noise.
Similarly, overfitting is a concern if we are modeling the proxy as a function of many observable indicators, or we do not have the domain knowledge we would need to impose a specific functional form for the relationship between these indicators and the outcome of interest.

Situations like these are common in political science, including the current context.
There are relatively few interstate disputes, and even fewer that involve just a single pair of states.
Even if we restrict ourselves to the National Material Capabilities data, there is an abundance of variables: six capability components for each side of the dispute, along with the six annual shares associated with each raw component, for a total of 24.
Incorporating time complicates matters even further.
As we do not have strong theory to guide us in choosing a functional form to relate capabilities to the probability of victory, we must instead take on our predictive approach.

The results from our predictive analysis thus map well to the analytic ideal of the probability of conflict in a hypothetical dispute between two states, which as we discuss above is an important variety of the more general concept of power.
We should note that our predictive analysis does not imply that we are producing a leader's subjective belief that she will prevail in said hypothetical dispute.
Instead, we produce the set of (objective) probabilities that best use the capabilities information at hand to predict the outcome.
We should also note that we are not forecasting, as we might if we made more explicit use of training and test sets defined by time, but instead are using all the data at once and assessing prediction via cross-validation.

%%% Local Variables:
%%% mode: latex
%%% TeX-master: "doe"
%%% End:
