% -*- mode: LaTeX; mode: flyspell; -*-
\documentclass[final]{beamer}
\mode<presentation> {
  \usetheme{Icy2}
}
\usepackage[english]{babel}
\usepackage[latin1]{inputenc}
\usepackage{amsmath}
\usepackage{amsthm}
\usepackage{amssymb}
\usepackage{mathtools}
\usepackage{calc}
\usepackage{multirow}
\usepackage{tikz}
\usefonttheme{serif}
\usepackage[orientation=landscape,size=a0,scale=1.4,debug]{beamerposter}

\title{Capability Ratios Predict Nothing: Introducing A Better Measure of Expected Dispute Outcomes}
% Author and institute are hard-coded into the template to avoid obnoxious
% issues with centering names with institutions, sorry
\date{August 29, 2014}

\setlength{\leftmargini}{125pt}

\newcommand{\pderiv}[2]{\frac{\partial #1}{\partial #2}}

\begin{document}

\begin{frame}{}

  \begin{columns}

    \begin{column}{.32\textwidth}
      % For why to use \vbox here, see
      % <http://tex.stackexchange.com/questions/15244/why-does-vfill-not-work-inside-a-beamer-column>
      \vbox to 0.95\textheight{%
        \begin{block}{Motivation}
          In bargaining models, states' expectations about who would win a
          potential dispute play a major role in international negotiations.

          \bigskip%
          Existing empirical work on conflict typically employs
          ratios of the Composite Index of National Capabilities to proxy for
          these expectations.

          \bigskip%
          Is this measure any good at actually predicting outcomes?
          If not, how can we do better?
        \end{block}

        \vfill

        \begin{block}{Data}
          \textbf{Sample:} Militarized international disputes, $N = 1{,}732$

          \bigskip%
          \textbf{Predictors:} Capability index components, with multiple
          imputation for missing values
          \begin{itemize}
            \item Iron and steel production
            \item Military expenditures
            \item Military personnel
            \item Primary energy consumption
            \item Total population
            \item Urban population
          \end{itemize}

          \bigskip%
          \textbf{Response:} Outcome --- ordered
          \begin{itemize}
            \item Side A Wins/Side B Yields
            \item Stalemate
            \item Side B Wins/Side A Yields
          \end{itemize}
        \end{block}

        \vfill

        \begin{block}{The Traditional Measure}
          \vspace{-1em}
          \begin{gather*}
            \text{CINC}_i
            =
            \frac{1}{6} \sum_{c=1}^6 \frac{
              \text{$i$'s level of component $c$}
            }{
              \text{total world level of component $c$}
            }
            \\[0.5em]
            \text{Capability Ratio}_{i, j}
            =
            \frac{\text{CINC}_i}{\text{CINC}_i + \text{CINC}_j}
          \end{gather*}

          \bigskip%
          We want to control, as best as possible, for expected dispute
          outcomes.  Does the Capability Ratio do that?
          
          \begin{itemize}
            \item Originally designed to measure power concentration at the
            system level (Singer et al 1972)

            \item Components not weighted

            \item Inflexible over time
          \end{itemize}
        \end{block}
      }
    \end{column}

    \begin{column}{.32\textwidth}
      \vbox to 0.95\textheight{%
        \begin{block}{Reasons for Skepticism about Capability Ratio}
          Results from ordered logit of Outcome on log(Capability Ratio):
          \begin{center}
            \includegraphics[scale=2]{fig/class-logit}%

            \vspace{-1em}%
            \includegraphics[scale=2]{fig/pp-logit}
          \end{center}
        \end{block}

        \vfill

        \begin{block}{Finding an Alternative}
          Can we predict dispute outcomes better without collecting new data?

          \bigskip%
          \alert{1.} Split data on CINC components and dispute outcomes into
          training set (80\%) and test set (20\%)

          \bigskip%
          \alert{2.} Within the training set:
          \begin{itemize}
            \item Use $k$-fold cross-validation to tune various models and
            estimate their ability to predict dispute outcomes out of sample
            
            \begin{itemize}
              \item Random forests (Breiman 2001)
              \item Support vector machines (Cortes and Vapnik 1995)
              \item Always include all six components for each side, with or
              without year of dispute start
            \end{itemize}

            \medskip%
            \item Choose the model with the best cross-validation performance,
            and estimate its parameters by fitting to the full training set
          \end{itemize}

          \bigskip%
          \alert{3.} Apply the fitted model to the test set to obtain an
          unbiased estimate of its out-of-sample error
        \end{block}
      }
    \end{column}

    \begin{column}{.32\textwidth} 
      \vbox to 0.95\textheight{%
        \begin{block}{Results}
          % Cross-validation within the training set suggests the best model is
          % a random forest with year of dispute included.
          \vspace{-1em}

          \begin{center}
            \includegraphics[scale=2]{fig/confusion-logit}%

            \includegraphics[scale=2]{fig/roc}
          \end{center}
        \end{block}

        \vfill

        \begin{block}{Conclusions}
          The Capability Ratio was not designed to predict dispute outcomes
          well, and indeed it does not.
          We construct a better measure using the same component data.

          \bigskip%
          We argue that proxy variables should be well-tuned for the use at
          hand and validated using out-of-sample predictive performance --- a
          lesson that applies elsewhere in IR and political science.
        \end{block}

        \vfill

        \begin{block}{References}
          \footnotesize

          Leo Breiman.  2001.  ``Random Forests.''  \textit{Machine Learning}
          45:5--32.

          \medskip%
          Corinna Cortes and Vladimir Vapnik.  1995.  ``Support-Vector
          Networks.''  \textit{Machine Learning} 20:273--297.

          \medskip%
          Douglas Mossman.  1999.  ``Three-Way ROCs.''  \textit{Medical
            Decision Making} 19:78--89.

          \medskip%
          J. David Singer, Stuart Bremer, and John Stuckey.  1972.
          ``Capability Distribution, Uncertainty, and Major Power War,
          1820--1965.''  In Bruce Russett (ed.), \textit{Peace, War, and
            Numbers}, Beverly Hills: Sage, 19--48.

          \medskip%
          R Packages: \texttt{Amelia}, \texttt{caret}, \texttt{ggplot2},
          \texttt{kernlab}, \texttt{MASS}, \texttt{randomForest}.
        \end{block}
      }
    \end{column}

  \end{columns}

\end{frame}

\end{document}
