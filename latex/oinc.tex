\documentclass[11pt,oneside]{article}

\usepackage[T1]{fontenc}
\usepackage[utf8]{inputenc}
\usepackage{sectsty}
\usepackage[labelsep=period,font=small,labelfont=bf]{caption}
\usepackage[charter]{mathdesign}
\usepackage[scaled]{helvet}
\usepackage[colorlinks=true,citecolor=blue]{hyperref}

% Section heading styling
\allsectionsfont{\sffamily}

\title{
  Capability Ratios Predict Nothing%
  % \thanks{%
  %   We have no one to thank yet.
  % }%
}
\author{%
  Robert J. Carroll%
  \thanks{%
    Rob's affiliation as of Polmeth goes here.
  }%
  \and%
  Brenton Kenkel%
  \thanks{
    Assistant Professor, Department of Political Science, Vanderbilt University.
    Email: \nolinkurl{brenton.kenkel@vanderbilt.edu}.
  }%
}

\begin{document}

\maketitle

\begin{abstract}
  Most empirical studies of international conflict use a ratio of military capability index scores as a proxy for the expected outcome of war.
  Despite the near-universal inclusion of capability ratios in statistical analyses of conflict, there has been no effort to validate whether these ratios are reliable predictors of war outcomes.
  Because states' expectations about potential war outcomes play a large role in theories of conflict, including the bargaining model of war, it is crucial for empirical research to control for a valid measure of these expectations.
  We proceed in three steps.
  First, we show that the predictive performance of the capability ratio is nil: it does no better than a null model at predicting war outcomes.
  Second, we advocate a data-driven approach to constructing a superior proxy for expected war outcomes, using the train-validate-test paradigm from the machine learning literature.
  Applying this method, we find that we can develop a substantially better measure of expected war outcomes using the same components that are used to construct the capability ratio.
  Third, we use our new measure to replicate existing studies of the role expectations play in the outbreak of international conflict.
\end{abstract}


\section{Introduction}


\section{Predictive Power and Proxy Variables}


\section{The Capability Ratio and Its Discontents}


\section{A Better Way to Measure Expected Dispute Outcomes}


\section{Measurement Validity and the Empirical Study of Conflict}


\section{Conclusion}


\end{document}
